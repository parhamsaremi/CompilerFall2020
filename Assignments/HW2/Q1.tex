%نام و نام خانوادگی:
%شماره دانشجویی: 
\مسئله{گرامر مبهم}
\پاسخ{

\begin{enumerate}
	\item
رشته‌ی \lr{00111} را در نظر می‌گیریم نشان می‌دهیم که این عبارت با دو درخت متفاوت پارس می‌شود. درخت اول به صورت زیر می‌باشد:
\begin {center}
\begin {tikzpicture}[-latex ,auto ,node distance =1 cm and 1cm ,on grid ,
semithick ,
state/.style ={ circle ,top color =white , bottom color = white ,
	draw,white , text=black , minimum width =1 cm}]
\node[state] (S1){$S$};
\node[state] (A1) [below left=of S1] {$A$};
\node[state] (S2) [below right =of S1] {$S$};
\node[state] (T1) [below left =of A1] {\lr{0}};
\node[state] (T2) [below right =of A1] {\lr{1}};
\node[state] (S3) [below right=of S2] {$\epsilon$};
\node[state] (A2) [below left=of T2] {$A$};
\node[state] (A3) [below left =of A2] {\lr{$A$}};
\node[state] (T4) [below right =of A2] {\lr{1}};
\node[state] (T5) [below right =of A3] {\lr{1}};
\node[state] (T6) [below left =of A3] {\lr{0}};



\path (S1) edge [] node[left] {} (A1);
\path (S1) edge [] node[] {} (S2);
\path (S2) edge [] node[] {} (S3);
\path (A1) edge [] node[] {} (T1);
\path (A1) edge [] node[] {} (T2);
\path (A1) edge [] node[] {} (A2);
\path (A2) edge [] node[] {} (A3);
\path (A2) edge [] node[] {} (T4);
\path (A3) edge [] node[] {} (T5);
\path (A3) edge [] node[] {} (T6);
\end{tikzpicture}
\end{center}
حال همین کار را با استفاده از یک درخت دیگر نیز انجام می‌دهیم:
\begin {center}
\begin {tikzpicture}[-latex ,auto ,node distance =1 cm and 1cm ,on grid ,
semithick ,
state/.style ={ circle ,top color =white , bottom color = white ,
	draw,white , text=black , minimum width =1 cm}]
\node[state] (S1){$S$};
\node[state] (S2) [below right =of S1] {$S$};
\node[state] (S3) [below right=of S2] {$\epsilon$};
\node[state] (A1) [below left=of S1] {$A$};
\node[state] (T0) [below right =of A1] {\lr{1}};
\node[state] (A2) [below left=of A1] {$A$};
\node[state] (T1) [below left =of A2] {\lr{0}};
\node[state] (T2) [below right =of A2] {\lr{1}};
\node[state] (A3) [below left =of T2] {\lr{$A$}};

\node[state] (T5) [below right =of A3] {\lr{1}};
\node[state] (T6) [below left =of A3] {\lr{0}};


\path (S1) edge [] node[] {} (A1);
\path (S1) edge [] node[] {} (S2);
\path (S2) edge [] node[] {} (S3);
\path (A1) edge [] node[] {} (T0);
\path (A1) edge [] node[] {} (A2);
\path (A2) edge [] node[] {} (A3);
\path (A2) edge [] node[] {} (T1);
\path (A2) edge [] node[] {} (T2);
\path (A3) edge [] node[] {} (T5);
\path (A3) edge [] node[] {} (T6);


\end{tikzpicture}
\end{center}
بنابراین با توجه به دو درخت بالا و آوردن مثال نقض ثابت می‌شود که گرامر نا‌مبهم نیست و مبهم است.
	\item
	از آنجایی که برای یک عبارت چند درخت پارس تولید می‌شود این مشکل به وجود می‌آید که کامپایلر نمی‌داند که باید از کدام استفاده کند و به عنوان مثال در صورت اهمیت اولویت در گرامر ممکن است که مفهوم‌های متفاوتی برداشت شود به عنوان مثال نیز در جزوه گرامری گفته شد که \lr{id+id*id} را به دو صورت متفاوت پارس می‌کرد و دچار مشکل می‌شود.
\end{enumerate}

}