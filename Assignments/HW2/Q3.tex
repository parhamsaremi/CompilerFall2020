%نام و نام خانوادگی:
%شماره دانشجویی: 
\مسئله{\lr{Recursive Descent}}
\پاسخ{
در این گرامر عبارت E دارد یک جمله شامل جمع و یا تفریق چند عبارت T می‌باشد (یا یک T تنها) بنابراین ابتدا برای حذف \lr{left recursion} گرامر را به صورت زیر تغییر می‌دهیم:
\begin{align*}
	\textcolor{red}{E} \rightarrow \textcolor{red}{T}\textcolor{blue}{+} \textcolor{red}{E}|\textcolor{red}{T}\textcolor{blue}{-}\textcolor{red}{E}|\textcolor{red}{T}
\end{align*}
حال با توجه به اینکه سه عبارت داریم که با T شروع می‌شوند با استفاده از \lr{factoring} گرامر به صورت زیر تبدیل می‌شود:

\begin{align*}
	\textcolor{red}{E} \rightarrow \textcolor{red}{T}\textcolor{red}{E'}\\
	\textcolor{red}{E'} \rightarrow \textcolor{blue}{+}\textcolor{red}{E}|\\
	\textcolor{blue}{-}\textcolor{red}{E}|\\
	\textcolor{blue}{\epsilon}
\end{align*}
حال به رفع مشکل عبارت T می‌پردازیم از آنجایی که نشان دهنده‌ی دنباله‌‌ای از دو عبارت \lr{10} یا \lr{11} می‌باشد،‌ گرامر زیر را جایگزین T می‌کنیم که نشان دهنده‌ی این نوع عبارات می‌باشد:
\begin{align*}
	\textcolor{red}{T} \rightarrow \textcolor{red}{B}\textcolor{red}{T}\\
	\textcolor{red}{B} \rightarrow \textcolor{blue}{1}\textcolor{red}{C}\\
	\textcolor{red}{C} \rightarrow \textcolor{blue}{1}|\textcolor{blue}{0}
\end{align*}
حال الگوریتم به صورت \lr{LL(1)} تبدیل شده است بنابراین شبه‌‌کدهای پایتون زیر را برای این گرامر‌ها استفاده می‌کنیم:
}
\begin{latin}
	\begin{verbatim}
		def parseE():
		   parseT()
		   parseEp()
		   if tokern == "\$":
		      return "success"
		   else:
		      return "error"
		
		
		def parseEp():
		   if token == "+":
		      token = nextToken()
		      parseE()
		   elif token == "-":
		      token = nextToken()
		      parseE()
		   elif token == "\$":
		      return
		   else:
		      return "error"
		
		def pareT():
		   parseB()
		   parseT()
		
		def parseB():
		   if token == "1":
		      token = nextToken()
		      parseC()
		   else:
		      return "error"
		
		def parseC():
		   if token == "1" or token == "0":
		      token = nextToken()
		      return
		   else:
		      return "error"
	\end{verbatim}
\end{latin}

