%نام و نام خانوادگی:
%شماره دانشجویی: 
\مسئله{طراحی \lr{DFA} }

\پاسخ{
برای دو قسمت اول بجز حالت شروع ۴ حالت تعریف شده که هر کدام نشان دهنده یکی از چهار حالت زوج-زوج، فرد-زوج، زوج-فرد، فرد-فرد می‌باشند و با تغییر حالت دادن بین این استیت‌ها با دریافت ۰ یا ۱ می‌توان فهمید که در حالت فعلی وضعیت زوج یا فرد بودن ۰ و ۱ ها به چه شکل می‌باشد.
\begin{enumerate}
	\item
	 همانند توضیحات آمده در بالا عمل می‌کنیم و همانطور که در شکل زیر مشخص می‌باشد $q_0$ که همان حالت شروع می‌باشد حال به ترتیب داریم.
	\begin{itemize}
		\item $q_1$ = 
		این حالت برای رشته‌هایی است که تعداد فرد ۰ دارند و تعداد زوج یک
		\item $q_2$ = 
		این حالت نشان‌دهنده‌ی زوج بودن ۰ و فرد بودن ۱ می‌باشد.
		\item $q_3$ = 
		این حالت نشان‌دهنده‌ی فرد بودن ۰ و ۱ می‌باشد
		\item $q_4$ = 
		این حالت نشان‌دهنده‌ی زوج بودن ۰ و ۱ می‌باشد که حالت پایانی نیز می‌باشد. 
	\end{itemize}
	حال با توجه به تعریف‌های بالا $DFA$ زیر گویا می‌باشد.

	\begin{latin}
		\begin{center}
			\begin{tikzpicture}[->,
				>=stealth,
				node distance=3cm,
				every state/.style={thick, fill=white!10},
				initial text=$ $,
				]
				\node[state, initial] (q0) {$q_0$};
				\node[state, above right of=q0](q1) {$q_1$};
				\node[state, below right of= q0] (q2) {$q_2$};
				\node[state, below right of=q1] (q3) {$q_3$};
				\node[state, accepting, right of=q3] (q4) {$q_4$};
				\draw 
				(q0) edge[above] node{0} (q1)
				(q0) edge[below] node{1} (q2)
				(q1) edge[above, bend left= 0.5cm] node{1} (q3)
				(q3) edge[below, bend left = 0.5cm] node{1} (q1)
				(q2) edge[below, bend right=0.5cm] node{0} (q3)
				(q3) edge[above, bend right=0.5cm] node{0} (q2)
				(q1) edge[above, bend left = 0.25cm] node{0} (q4)
				(q4) edge[above, bend right = 1cm] node{0} (q1)
				(q2) edge[below, bend right = 0.25cm] node{1} (q4)
				(q4) edge[below, bend left = 1cm] node{1} (q2)
				;
			\end{tikzpicture}
		\end{center}
	\end{latin}
	\item
	مانند قسمت قبل می‌باشد فقط حالت‌های نهایی آن متفاوت با اولی می‌باشد و  با توجه به تعاریف حالت‌ها در بخش قبلی باید $q_1$ و $q_2$ حالت‌های پایانی باشند. داریم:
	\begin{latin}
		\begin{center}
			\begin{tikzpicture}[->,
				>=stealth,
				node distance=3cm,
				every state/.style={thick, fill=white!10},
				initial text=$ $,
				]
				\node[state, initial] (q0) {$q_0$};
				\node[state, accepting, above right of=q0](q1) {$q_1$};
				\node[state, accepting, below right of= q0] (q2) {$q_2$};
				\node[state, below right of=q1] (q3) {$q_3$};
				\node[state, right of=q3] (q4) {$q_4$};
				\draw 
				(q0) edge[above] node{0} (q1)
				(q0) edge[below] node{1} (q2)
				(q1) edge[above, bend left= 0.5cm] node{1} (q3)
				(q3) edge[below, bend left = 0.5cm] node{1} (q1)
				(q2) edge[below, bend right=0.5cm] node{0} (q3)
				(q3) edge[above, bend right=0.5cm] node{0} (q2)
				(q1) edge[above, bend left = 0.25cm] node{0} (q4)
				(q4) edge[above, bend right = 1cm] node{0} (q1)
				(q2) edge[below, bend right = 0.25cm] node{1} (q4)
				(q4) edge[below, bend left = 1cm] node{1} (q2)
				;
			\end{tikzpicture}
		\end{center}
	\end{latin}
	\item
	این قسمت با استفاده از یک حالت $Garbage$ که با $G$ مشخص شده و هفت حالت اصلی از $q_0$ تا $q_6$ حل شده که $DFA$ آن در زیر آمده است:
	\begin{latin}
		\begin{center}
			\begin{tikzpicture}[->,
				>=stealth,
				node distance=2cm,
				every state/.style={thick, fill=white!10},
				initial text=$ $,
				]
				\node[state, initial] (q0) {$q_0$};
				\node[state, right of=q0](q1) {$q_1$};
				\node[state, right of= q1] (q2) {$q_2$};
				\node[state, above right of=q2] (q3) {$q_3$};
				\node[state, accepting,below right of=q2] (q4) {$q_4$};
				\node[state, right of=q3] (q5) {$q_5$};
				\node[state, accepting, right of=q5] (q6) {$q_6$};
				\node[state, below of= q4] (G) {$G$}; 
				\draw 
				(q0) edge[above] node{0} (q1)
				(q1) edge[below] node{1} (q2)
				(q1) edge[above, bend left = 0.5cm] node{0} (q3)
				(q2) edge[above] node{0} (q3)
				(q2) edge[below] node{1} (q4)
				(q3) edge[above] node{1} (q5)
				(q4) edge[right] node{0} (q3)
				(q5) edge[above] node{1} (q6)
				
				(q4) edge[left] node{1} (G)
				(q0) edge[below] node{1} (G)
				(q3) edge[right, bend left = 1.1cm] node{0} (G)
				(q5) edge[right, bend left=0.5cm] node{0} (G)
				(q6) edge[right, bend left = 0.125cm] node{0,1} (G)
				(G) edge[below, loop below] node{0,1} (G)
				;
			\end{tikzpicture}
		\end{center}
	\end{latin}
\end{enumerate}

}
