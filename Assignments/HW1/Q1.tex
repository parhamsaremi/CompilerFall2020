%نام و نام خانوادگی:
%شماره دانشجویی: 
\مسئله{نوشتن عبارات منظم}

\پاسخ{}
\\
۱) برای حل این قسمت باید حالت‌بندی بکنیم. به این صورت که قسمت عددی نمایش علمی، یا بزرگ‌تر مساوی ۱ و کوچک‌تر از ۵ است و یا بزرگ‌تر مساوی ۵ و کوچک‌تر از ۱۰ است.
\\
اگر حالت اول رخ دهد، توان باید عددی صحیح و کم‌تر از ۲ باشد. اگر حالت دوم رخ دهد، توان باید عدد صحیحی نامثبت باشد.
\\
حال شروع به بدست آوردن regex برای حالت اول می‌کنیم. قسمت عددی می‌تواند مثبت یا منفی باشد و ممکن است ممیز هم داشته باشد، اما با توجه به محدوده حالت اول، قسمت صحیح آن برابر با ۱ تا ۴ است. پس regex آن به صورت زیر بدست می‌آید:
\begin{latin}
	\verb/[-+]?[1-4](\.[0-9]+)?/
\end{latin}
اگر علامت توان مثبت باشد، مقدار عددی آن برابر با ۰ یا ۱ است. اگر منفی باشد می‌تواند هرمقداری داشته باشد. پس با توجه به این موضوع regex توان به صورت زیر بدست می‌آید:
\begin{latin}
	\verb/[eE](([-+]?[0-1])|(-[0-9]+))/
\end{latin}
پس در کل regex برای حالت اول بدون در نظر گرفتن space های اول و آخر آن، به صورت زیر بدست می‌آید:
\begin{latin}
	\verb/([-+]?[1-4](\.[0-9]+)?([eE](([-+]?[0-1])|(-[0-9]+)))?/
\end{latin}
به طور مشابه regex را برای حالت دوم نیز بدست می‌آوریم. دقت شود که در این حالت، در قسمت عددی نمایش علمی، قسمت صحیح مقداری از ۵ تا ۹ است و توان نمی‌تواند مثبت باشد. پس regex می‌شود:
\begin{latin}
	\verb/[-+]?[5-9](\.[0-9]+)?([eE](([-+]?0)|(-[0-9]+)))?)/
\end{latin}
حال برای بدست آوردن regex کلی کافی است تا بین این دو یک or قرار دهیم و البته space ها را هم در ابتدا و انتها چک کنیم (به علت طولانی شدن در دو خط نمایش داده شده است):
\begin{latin}
	\verb/ *(([-+]?[1-4](\.[0-9]+)?([eE](([-+]?[0-1])|(-[0-9]+)))?)/
	\linebreak
	\verb/|([-+]?[5-9](\.[0-9]+)?([eE](([-+]?0)|(-[0-9]+)))?)) */
\end{latin}
۲) جواب به صورت زیر می‌شود. دقت شود که $\textbackslash1$ به این معناست که باید به جای آن دقیقا همان عبارت group اول در regex تکرار شود که یعنی عبارت داخل پرانتز:
\begin{latin}
	\verb/021([0-9])\1[0-9]{6}/
\end{latin}
۳) جواب به صورت زیر می‌شود:
\begin{latin}
	\verb/[a-z]+\.[0-9]{3}\.[A-Z]\.([0-9])\1/
\end{latin}
