%نام و نام خانوادگی:
%شماره دانشجویی:
\مسئله{محاسبه‌ی اعداد}

\پاسخ
\\
{برای هر استیت مجموعه حالات متناظر با آن در NFA را مینویسیم.
همچنین حالاتی که در DFAهر بیان نشده‌اند، حالت اضافی هستند که اکسپت نیز نخواهند شد و از آوردن آن‌ها برای جلوگیری از شلوغی شکل پرهیز شده است. برای این موارد فرض شده است که استیت خالی‌ای  موجود است و حالات نامطلوب به آن استیت میروند و در آن نیز می‌مانند.
\\
1.
\begin{latin}
	\begin{itemize}
		\item
		A:{{$q_1$} , {$q_0$} }
		\item
		B:{ {$q_1$} }
		\item
		C:{ {$q_2$} }
	\end{itemize}
\end{latin}
\begin{center}
\begin{tikzpicture}[shorten >=1pt,node distance=2cm,on grid,auto]
\node[state,initial,accepting]   (A)                    {$A$};
\node[state,accepting] (B)  [above right of=A]    {$B$};
\node[state]           (C)  [below right of=B]    {$C$};
\path[->]
(A) edge [bend left]  node {a}    (B)
(B) edge [bend left]  node {b}    (C)
 edge [loop above] node {a}    (   )
(C) edge [bend left]  node {c}  (A)
	edge [loop right] node {b}    (   );
\end{tikzpicture}
\end{center}

2.
\begin{latin}
	\begin{itemize}
	\item
 	A:{ {$q_0$} }
    \item
    B:{ {$q_1$} }
    \item
    C:{ {$q_2$} }
    \item
    D:{ {$q_3$} }
\end{itemize}
\end{latin}

\begin{center}
\begin{tikzpicture}[shorten >=1pt,node distance=2cm,on grid,auto]
\node[state,initial]   (A)                    {$A$};
\node[state] (B)  [above right of=A]    {$B$};
\node[state]           (C)  [below right of=B]    {$C$};
\node[state,accepting]           (D)  [below right of=A]    {$D$};
\path[->]
(A) edge [bend left]  node {a}    (B)
(B) edge [bend left]  node {b}    (C)
	edge []  node {a}    (A)
(C) edge []  node {b}  (B)
 	edge [bend left]  node {c}  (D)
(D) edge   node {a}  (B)
	edge [bend left]  node {c}  (C);
\end{tikzpicture}
\end{center}
3.
\begin{latin}
	\begin{itemize}
	\item
    A:{ {$s_0$} }
    \item
    B:{ {$s_3$} , {$s_2$} , {$s_1$}}
    \item
    C:{ {$q_3$} }
	\end{itemize}
\end{latin}
\begin{center}
\begin{tikzpicture}[shorten >=1pt,node distance=2cm,on grid,auto]

\node[state,initial]   (A)                    {$A$};
\node[state,accepting] (B)  [above right of=A]    {$B$};
\node[state,accepting]           (C)  [below right of=B]    {$C$};
\path[->]
(A) edge [bend left]  node {a}    (B)
(B) edge [loop above]  node {a}    ()
 	edge [bend left]  node {b,c}    (C)
(C) edge [loop right] node {c}    (   );

\end{tikzpicture}
\end{center}
4.
\begin{latin}
	\begin{itemize}
		\item
		A:{ {$q_0$} }
		\item
		B:{ {$q_1$} , {$q_0$} }
		\item
		C:{ {$q_1$} , {$q_0$} }
		\item
		D:{ {$q_2$},{$q_1$},{$q_0$} }
		\item
		E:{ {$q_3$},{$q_2$},{$q_1$} }
		\item
		F:{ {$q_3$}, {$q_2$},{$q_1$},{$q_0$} }
		\item
		G:{ {$q_3$} , {$q_2$} }
		\item
		H:{ {$q_3$}}
	\end{itemize}
\end{latin}


\begin{center}
\begin{tikzpicture}[shorten >=1pt,node distance=2cm,on grid,auto]
\node[state,initial]   (A)                    {$A$};
\node[state] (B)  [right of=A]    {$B$};
\node[state]           (C)  [right of=B]    {$C$};
\node[state]   (D)         [right of=C]           {$D$};
\node[state,accepting] (E)  [ below of = C]    {$E$};
\node[state,accepting]           (F)  [right of=D]    {$F$};
\node[state,accepting]   (G)        [right of=F]            {$G$};
\node[state,accepting] (H)  [right of=G]    {$H$};
\path[->]
(A) edge   node {a}    (B)
    edge [bend left]  node {b}    (C)
(B) edge [bend right]  node {b}    (E)
 	edge [bend left]  node {a}    (D)
(C) edge [bend left]  node {a}    (E)
    edge [bend left]  node {b}    (G)
(D) edge [bend left]  node {a}    (F)
    edge [bend left]  node {b}    (E)
(E) edge [bend right]  node {b}    (G)
    edge [loop below]  node {a}    ()
(F) edge [bend left]  node {b}    (E)
    edge [loop above]  node {a}    ()
(G) edge [bend left]  node {b}    (H)
    edge [loop below]  node {a}    ()
(H) edge [loop below]  node {a}    ();

\end{tikzpicture}
\end{center}