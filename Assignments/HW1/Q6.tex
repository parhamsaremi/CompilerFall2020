%نام و نام خانوادگی:
%شماره دانشجویی: 
\مسئله{طراحی واژه‌یاب}

\پاسخ{}
\\

۱) DFA
 قسمت‌های بخش‌های مختلف سوال را رسم می‌کنم و در نهایت طریقه ادغام آن‌ها را شرح می‌دهم. دقت شود که در DFA های زیر، استیت start درواقع اولین استیت DFA کلی است که در ابتدا روی آن قرار داریم.
\\\\
 الف) کامنت تک‌خطی و چندخطی:
 \begin{latin}
 	\begin{center}
 		\begin{tikzpicture}[->,
 			>=stealth,
 			node distance=3cm,
 			every state/.style={thick, fill=white!10},
 			initial text=$ $,
 			]
 			\node[state, initial] (q0) {start};
 			\node[state, right of=q0] (q1) {};
 			\node[state, right of=q1](q2) {};
 			\node[state, right of= q2] (q3) {};
 			\node[state, below right of=q1] (q4) {};
 			\node[state, accepting, below of=q4] (q5) {one line};
 			\node[state, accepting, above right of=q2] (q6) {multi line};
 			\draw 
 			(q0) edge[above] node{\#} (q1)
 			(q1) edge[above] node{(} (q2)
 			(q2) edge[above, bend left] node{\textbackslash} (q3)
 			(q3) edge[below, bend left] node{$\sum-\textbackslash$} (q2)
 			(q2) edge[loop above] node{$\sum-\textbackslash-)$} (q2)
 			(q2) edge[above right, above] node{)} (q6)
 			(q1) edge[below right, right] node{$\sum-\textbackslash n-($} (q4)
 			(q4) edge[right] node{$\textbackslash n$} (q5)
 			(q4) edge[loop right] node{$\sum-\textbackslash n$} (q4)
 			(q1) edge[below, left, bend right = 25cm] node{$\textbackslash n$} (q5)
 			;
 		\end{tikzpicture}
 	\end{center}
 \end{latin}
ب) اعداد (مبنای ۱۰، ۸، ۱۶، نمایش علمی):
\begin{latin}
	\begin{center}
		\begin{tikzpicture}[->,
			>=stealth,
			node distance=3cm,
			every state/.style={thick, fill=white!10},
			initial text=$ $,
			]
			\node[state, initial] (q0) {start};
			\node[state, accepting ,above right of=q0] (q1) {octal};
			\node[state, accepting, right of=q0](q2) {dec};
			\node[state, below right of= q0] (q3) {};
			\node[state, accepting, right of= q2] (q4) {dec};
			\node[state, accepting, below right of= q3] (q5) {octal};
			\node[state, right of= q5] (q6) {};
			\node[state, accepting, below right of= q5] (q7) {octal};
			\node[state, right of= q1] (q8) {};
			\node[state, accepting, above right of= q1] (q9) {};
			\node[state, accepting, above of= q1] (q10) {octal};
			\node[state, accepting, right of= q6] (q11) {hex};
			\node[state, right of= q8] (q12) {};
			\node[state, right of= q12] (q13) {};
			\node[state, accepting, right of= q13] (q14) {scientific};
			\node[state, accepting, right of= q9] (q15) {hex};
			\draw 
			(q0) edge[above right, above] node{0} (q1)
			(q0) edge[right, above] node{[1-9]} (q2)
			(q0) edge[below right, above] node{[-+]} (q3)
			(q2) edge[right, above] node{[0-9]} (q4)
			(q3) edge[above right, above] node{[1-9]} (q4)
			(q3) edge[below right, above] node{0} (q5)
			(q5) edge[right, above] node{[xX]} (q6)
			(q5) edge[below right, above] node{[0-7]} (q7)
			(q1) edge[right, above] node{\textbackslash.} (q8)
			(q1) edge[above right, above] node{[xX]} (q9)
			(q1) edge[above, left] node{[0-7]} (q10)
			(q2) edge[above right, above] node{\textbackslash.} (q8)
			(q10) edge[loop above, above] node{[0-7]} (q10)
			(q4) edge[loop right, right] node{[0-9]} (q4)
			(q6) edge[right, above] node{[0-9a-fA-F]} (q11)
			(q11) edge[loop above, above] node{[0-9a-fA-F]} (q11)
			(q7) edge[loop right, above] node{[0-7]} (q7)
			(q8) edge[right, above] node{[0-9]} (q12)
			(q12) edge[right, above] node{e} (q13)
			(q13) edge[right, above] node{[0-9]} (q14)
			(q14) edge[loop above, above] node{[0-9]} (q14)
			(q12) edge[loop above, above] node{[0-9]} (q12)
			(q9) edge[right, above] node{[0-9a-fA-F]} (q15)
			(q15) edge[loop above, above] node{[0-9a-fA-F]} (q15)
			;
		\end{tikzpicture}
	\end{center}
\end{latin}
ج) رشته‌ها، شناسه‌ها و کلیدواژه‌های for و foreach : 
\\
دقت شود به علت جلوگیری از شلوغ شدن بعضی یال‌ها در شکل زیر رسم نشده‌اند و اینجا آن‌ها را توضیح می‌دهم.
\\
همانند ۲ یالی که از استیت شماره ۳ به استیت‌های ۱ و ۲ رسم کردم، باید برای بقیه استیت‌های پایین استیت شماره ۳ هم کشید. با این تفاوت که یال استیت ۴ به ۱ به جای 
$[a-zA-Z0-9]-o$
 باید
 $[a-zA-Z0-9]-r$
 باشد. یال underscore تغییری نمی‌کند.
 \\
 این یال‌ها به این خاطر هستند که اگر در حین تشخیص for یا foreach ناگهان به حرف اشتباه رسیدیم، وارد بخش شناسه‌ها شویم. 
\begin{latin}
	\begin{center}
		\begin{tikzpicture}[->,
			>=stealth,
			node distance=3cm,
			every state/.style={thick, fill=white!10},
			initial text=$ $,
			]
			\node[state, initial] (q0) {start};
			\node[state, below left of=q0] (q1) {};
			\node[state, below of=q1] (q2) {};
			\node[state, below of= q2] (q3) {};
			\node[state, below of=q3] (q4) {};
			\node[state, left of=q4] (q5) {};
			\node[state, below of=q4] (q6) {};
			\node[state, below of=q6] (q7) {};
			\node[state, accepting, below of=q7] (q8) {string};
			\node[state, accepting, right of=q2] (q9) {1};
			\node[state, accepting, below of=q9] (q10) {2};
			\node[state, right of=q9] (q11) {3};
			\node[state, below of=q11] (q12) {4};
			\node[state, accepting, below of=q12] (q13) {for};
			\node[state, below of=q13] (q14) {};
			\node[state, below of=q14] (q15) {};
			\node[state, below of=q15] (q16) {};
			\node[state, accepting, below of=q16] (q17) {foreach};
			\draw 
			(q0) edge[right, left] node{r} (q1)
			(q1) edge[right, left] node{"} (q2)
			(q2) edge[right, left] node{"} (q3)
			(q3) edge[right, left] node{"} (q4)
			(q4) edge[above, below, bend left] node{\textbackslash} (q5)
			(q5) edge[below, above, bend left] node{$\sum-\textbackslash$} (q4)
			(q4) edge[loop right, right] node{$\sum-\textbackslash-"$} (q4)
			(q5) edge[loop left, left] node{\textbackslash} (q5)
			(q4) edge[below, right, bend left] node{"} (q6)
			(q6) edge[above, left, bend left] node{$\sum-"$} (q4)
			(q6) edge[left, left] node{"} (q7)
			(q7) edge[right, left, bend left] node{$\sum-"$} (q4)
			(q7) edge[below, left] node{"} (q8)
			(q0) edge[below right, above] node{f} (q11)
			(q11) edge[below, right] node{o} (q12)
			(q12) edge[below, right] node{r} (q13)
			(q13) edge[below, right] node{e} (q14)
			(q14) edge[below, right] node{a} (q15)
			(q15) edge[below, right] node{c} (q16)
			(q16) edge[below, right] node{h} (q17)
			(q0) edge[below right, left] node{[a-zA-Z0-9]-r-f} (q9)
			(q9) edge[loop right, above] node{[a-zA-Z0-9]} (q9)
			(q9) edge[below, right, bend left] node{\_} (q10)
			(q10) edge[above, left, bend left] node{[a-zA-Z0-9]} (q9)
			(q11) edge[left, below, bend left] node{[a-zA-Z0-9]-o} (q9)
			(q11) edge[below left, below, bend left] node{\_} (q10)
			
			;
		\end{tikzpicture}
	\end{center}
\end{latin}
حال DFA ها را ادغام می‌کنیم. کافی است استیت start که در همه مشترک هست را یه یک استیت یکسان تبدیل کنیم و با توجه به کاراکتری که می‌بینیم وارد یکی از ۳ DFA بالا می‌شویم.


۲) کد به زبان پایتون در زیر آمده است:
\begin{latin}
	\begin{verbatim}
		import re   
		
		while True:
		   s = input()
		   for i in re.finditer(
		      "((-|\+)?0(x|X)([0-9a-fA-F]+))|((-|\+)?(0\d*))|((-|\+)?[1-9]\d+)",s):
		      a = i.group(0)
		      if re.match("^(-|\+)?(0\d*)$",a):
		         print(int(a,8), "octate")
		      if re.match("^(-|\+)?0(x|X)([0-9a-fA-F]+)$",a):
		         print((int(a,0)), "hex")
		      if re.match("^(-|\+)?[1-9]\d+$",a):
		         print((int(a)), "decimal")
	\end{verbatim}
\end{latin}