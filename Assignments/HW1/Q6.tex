%نام و نام خانوادگی:
%شماره دانشجویی: 
\مسئله{طراحی واژه‌یاب}

\پاسخ{}
\\

۱) DFA
 قسمت‌های بخش‌های مختلف سوال را رسم می‌کنم و در نهایت طریف ادغام آن‌ها را شرح می‌دهم. دقت شود که در DFA های زیر، استیت start درواقع اولین استیت DFA کلی است که در ابتدا روی آن قرار داریم.
\\\\
 الف و ب) کامنت تک‌خطی و چندخطی:
 \begin{latin}
 	\begin{center}
 		\begin{tikzpicture}[->,
 			>=stealth,
 			node distance=3cm,
 			every state/.style={thick, fill=white!10},
 			initial text=$ $,
 			]
 			\node[state, initial] (q0) {start};
 			\node[state, right of=q0] (q1) {};
 			\node[state, right of=q1](q2) {};
 			\node[state, right of= q2] (q3) {};
 			\node[state, below right of=q1] (q4) {};
 			\node[state, accepting, below of=q4] (q5) {};
 			\node[state, accepting, above right of=q2] (q6) {};
 			\draw 
 			(q0) edge[above] node{\#} (q1)
 			(q1) edge[above] node{(} (q2)
 			(q2) edge[above, bend left] node{\textbackslash} (q3)
 			(q3) edge[below, bend left] node{$\sum-\textbackslash$} (q2)
 			(q2) edge[loop above] node{$\sum-\textbackslash-)$} (q2)
 			(q2) edge[above right, above] node{)} (q6)
 			(q1) edge[below right, right] node{$\sum-\textbackslash n-($} (q4)
 			(q4) edge[right] node{$\textbackslash n$} (q5)
 			(q4) edge[loop right] node{$\sum-\textbackslash n$} (q4)
 			(q1) edge[below, left, bend right = 25cm] node{$\textbackslash n$} (q5)
 			;
 		\end{tikzpicture}
 	\end{center}
 \end{latin}
ج) اعداد صحیح:
 \begin{latin}
	\begin{center}
		\begin{tikzpicture}[->,
			>=stealth,
			node distance=3cm,
			every state/.style={thick, fill=white!10},
			initial text=$ $,
			]
			\node[state, initial] (q0) {start};
			\node[state, accepting, above right of=q0] (q1) {W};
			\node[state, below right of=q0](q2) {X};
			\node[state, accepting, right of= q2] (q3) {};
			\draw 
			(q0) edge[above] node{[1-9]} (q1)
			(q0) edge[below] node{[-+]} (q2)
			(q1) edge[loop above] node{[0-9]} (q1)
			(q2) edge[right,above] node{[1-9]} (q3)
			(q3) edge[loop above] node{[0-9]} (q3)
			;
		\end{tikzpicture}
	\end{center}
\end{latin}
د) اعداد صحیح مبنای ۸:
\begin{latin}
	\begin{center}
		\begin{tikzpicture}[->,
			>=stealth,
			node distance=3cm,
			every state/.style={thick, fill=white!10},
			initial text=$ $,
			]
			\node[state, initial] (q0) {start};
			\node[state, accepting, above right of=q0] (q1) {Z};
			\node[state, below right of=q0] (q2) {X};
			\node[state, accepting, right of= q2] (q3) {Y};
			\draw 
			(q0) edge[above right, above] node{0} (q1)
			(q0) edge[below right, above] node{[-+]} (q2)
			(q2) edge[right, above] node{0} (q3)
			(q1) edge[loop below] node{[0-7]} (q1)
			(q4) edge[loop below] node{[0-7]} (q3)
			;
		\end{tikzpicture}
	\end{center}
\end{latin}
ه) اعداد صحیح مبنای ۱۶:
 \begin{latin}
	\begin{center}
		\begin{tikzpicture}[->,
			>=stealth,
			node distance=3cm,
			every state/.style={thick, fill=white!10},
			initial text=$ $,
			]
			\node[state, initial] (q0) {start};
			\node[state, above right of=q0] (q1) {X};
			\node[state, right of=q1] (q2) {Y};
			\node[state, accepting, right of= q2] (q3) {};
			\node[state, below right of=q0] (q4) {Z};
			\node[state, accepting, right of=q4] (q5) {};
			\draw 
			(q0) edge[above right, above] node{[-+]} (q1)
			(q1) edge[right, above] node{0} (q2)
			(q2) edge[right, above] node{[xX]} (q3)
			(q0) edge[below right, above] node{0} (q4)
			(q4) edge[right, above] node{[xX]} (q5)
			(q3) edge[loop below] node{[0-9a-fA-F]} (q3)
			(q5) edge[loop below] node{[0-9a-fA-F]} (q5)
			;
		\end{tikzpicture}
	\end{center}
\end{latin}
و) رشته‌ها:
 \begin{latin}
	\begin{center}
		\begin{tikzpicture}[->,
			>=stealth,
			node distance=3cm,
			every state/.style={thick, fill=white!10},
			initial text=$ $,
			]
			\node[state, initial] (q0) {start};
			\node[state, right of=q0] (q1) {1};
			\node[state, right of=q1] (q2) {};
			\node[state, right of= q2] (q3) {};
			\node[state, right of=q3] (q4) {};
			\node[state, above of=q4] (q5) {};
			\node[state, below of=q4] (q6) {};
			\node[state, left of=q6] (q7) {};
			\node[state, accepting, left of=q7] (q8) {};
			\draw 
			(q0) edge[right, above] node{r} (q1)
			(q1) edge[right, above] node{"} (q2)
			(q2) edge[right, above] node{"} (q3)
			(q3) edge[right, above] node{"} (q4)
			(q4) edge[above, right] node{\textbackslash} (q5)
			(q5) edge[below, left] node{$\sum-\textbackslash$} (q4)
			(q4) edge[loop right, right] node{$\sum-\textbackslash-"$} (q4)
			(q5) edge[loop above, above] node{\textbackslash} (q5)
			(q4) edge[below, right, bend left] node{"} (q6)
			(q6) edge[above, left, bend left] node{$\sum-"$} (q4)
			(q6) edge[left, above] node{"} (q7)
			(q7) edge[right, above, bend left] node{$\sum-"$} (q4)
			(q7) edge[left, above] node{"} (q8)
			;
		\end{tikzpicture}
	\end{center}
\end{latin}
ز) شناسه‌ها: (به این علت دو حرف f و r را در ابتدا حذف کردم که f حرف اول کلیدواژه‌های for و foreach است و r هم حرف اول رشته‌ها است و برای حل این مشکل باید در نهایت از DFA های این دو گروه به DFA شناسه‌ها با یال‌هایی که در ادامه طراحی می‌کنم بازگردم.)
\begin{latin}
	\begin{center}
		\begin{tikzpicture}[->,
			>=stealth,
			node distance=3cm,
			every state/.style={thick, fill=white!10},
			initial text=$ $,
			]
			\node[state, initial] (q0) {start};
			\node[state, accepting, right of=q0] (q1) {1};
			\node[state, accepting, right of=q1] (q2) {2};
			\draw 
			(q0) edge[right, above] node{[a-zA-Z]-f-r} (q1)
			(q1) edge[above right, bend left, above] node{\_} (q2)
			(q2) edge[below left, bend left, below] node{[a-zA-Z0-9]} (q1)
			(q1) edge[loop below, below] node{[a-zA-Z0-9]} (q1)
			;
		\end{tikzpicture}
	\end{center}
\end{latin}
ح) اعداد حقیقی مبنای ۱۰ با نمایش علمی:
\begin{latin}
	\begin{center}
		\begin{tikzpicture}[->,
			>=stealth,
			node distance=3cm,
			every state/.style={thick, fill=white!10},
			initial text=$ $,
			]
			\node[state, initial] (q0) {start};
			\node[state, above right of=q0] (q1) {Z};
			\node[state, below right of=q0] (q2) {W};
			\node[state, above right of= q2] (q3) {};
			\node[state, right of=q3] (q4) {};
			\node[state, right of=q4] (q5) {};
			\node[state, accepting, right of=q5] (q6) {};
			\draw 
			(q0) edge[above right, above] node{0} (q1)
			(q0) edge[below right, above right] node{[1-9]} (q2)
			(q1) edge[below right, above] node{\textbackslash.} (q3)
			(q2) edge[above right, above] node{\textbackslash.} (q3)
			(q3) edge[right, above] node{[0-9]} (q4)
			(q4) edge[loop above, above] node{[0-9]} (q4)
			(q4) edge[right, above] node{e} (q5)
			(q5) edge[right, above] node{[0-9]} (q6)
			(q6) edge[loop above] node{[0-9]} (q6)
			;
		\end{tikzpicture}
	\end{center}
\end{latin}
ط) کلیدواژه‌های for و foreach :
\begin{latin}
	\begin{center}
		\begin{tikzpicture}[->,
			>=stealth,
			node distance=2cm,
			every state/.style={thick, fill=white!10},
			initial text=$ $,
			]
			\node[state, initial] (q0) {start};
			\node[state, right of=q0] (q1) {1};
			\node[state, right of=q1] (q2) {};
			\node[state, accepting, right of= q2] (q3) {};
			\node[state, right of=q3] (q4) {};
			\node[state, right of=q4] (q5) {};
			\node[state, right of=q5] (q6) {};
			\node[state, accepting, right of=q6] (q7) {};
			\draw 
			(q0) edge[right, above] node{f} (q1)
			(q1) edge[right, above] node{o} (q2)
			(q2) edge[right, above] node{r} (q3)
			(q3) edge[right, above] node{e} (q4)
			(q4) edge[right, above] node{a} (q5)
			(q5) edge[right, above] node{c} (q6)
			(q6) edge[right, above] node{h} (q7)
			
			;
		\end{tikzpicture}
	\end{center}
\end{latin}
حال DFA ها را ادغام می‌کنیم. توجه شود در DFA های بالا، استیت‌هایی که درون آن‌ها حرف یکسان نوشته شده است، بدان معناست که آن استیت در DFA نهایی باید به دقیقا یک استیت تبدیل شود. مثلا در DFA های
\underline{اعداد صحیح}
و
\underline{نمایش علمی}
استیت W مشترک است و باید به یک استیت تبدیل شود چرا که در هر دو DFA با کاراکتر $[1-9]$ وارد آن می‌شویم و نمی‌توانند دو استیت متفاوت باشند.
\\\\
ابتدا سه DFA 
\underline{شناسه‌ها‌}
،
\underline{for/foreach}
و 
\underline{رشته‌ها}
 را ادغام می‌کنیم. برای ادغام این ۳ به این صورت عمل می‌کنیم که دقیقا DFA های طراحی شده در بخش‌های قبل را کنار هم می‌گذاریم، و صرفا یک سری یال بین آن‌ها اضافه می‌کنیم. این یال‌ها به این خاطر است که بتوانیم به عنوان مثال عبارتی مثل fob را یک شناسه تشخیص دهیم. اگر یالی اضافه نکنیم امکان تشخیص چنین عباراتی را نخواهیم داشت.
\\
یال‌های اضافه شده از این قرار خواهند بود. یک یال از استیت شماره ۱ DFA رشته‌ها (در شکل مشخص شده) به استیت شماره ۱ DFA شناسه‌ها در همین جهت وصل می‌کنیم که کاراکتر مورد قبول آن یکی از کاراکترهای $[a-zA-Z0-9]$ است. هم‌چنین یک یال دیگر از استیت شماره ۱ رشته‌ها به استیت شماره ۲ شناسه وصل می‌کنیم که کاراکتر مورد قبول آن 
\verb/_/
(underscore) است.
\\
 حال یال‌های بین for/foreach و شناسه‌ها را اضافه می‌کنیم. یال‌ها به این صورت خواهند بود که برای هرکدام از استیت‌های for/foreach، اگر به کاراکتری برخوردیم که جزو [a-zA-Z0-9] این بود ولی برابر با کاراکتر منتهی به استیت بعدی در for/foreach نبود، به استیت شماره ۱ DFA شناسه‌ها می‌رویم. هم‌چنین اگر به کاراکتر underscore برخوردیم به استیت شماره ۲ شناسه‌ها می‌رویم. درواقع با این کار در هر لحظه که عبارت توانایی تطابق با کلیدواژه‌های for یا foreach را از دست بدهد، به DFA شناسه‌ها منتقل می‌شویم. برای فهم راحت‌تر یکی از یال‌ها را مثال می‌زنم. به عنوان مثال باید یک یال از استیت شماره ۱ for/foreach به استیت شماره ۱ شناسه‌ها وصل شود که کاراکترهای
$[a-zA-Z0-9]-o$
 را قبول می‌کند. هم‌چنین اگر کاراکتر underscore دیدیم باید به استیت شماره ۲ شناسه‌ها برویم.
 \\
 روند ادغام کردن بقیه DFA ها ساده است. کافی است آن‌ها را در کنار هم بگذاریم، چرا که ارتباطی با یکدیگر ندارند و به سادگی ادغام می‌شوند. صرفا باید دقت شود که استیت‌های هم‌نام به یک استیت تبدیل می‌شوند ولی بقیه استیت‌ها هرکدام به طور یکتا به یک استیت جدید تبدیل می‌شوند.
 \\
 در مورد استیت های پایانی هم توجه شود که اگر در یک DFA مشخص (مثلا اعداد مبنای ۸) باشیم و در یکی از استیت‌های پایانی ورودی تمام شود و به آخر آن برسیم، DFA عبارت خوانده شده تا حالا را به عنوان عدد مبنای ۸ برمی‌گرداند.
 \\
 هم‌چنین اگر در هر لحظه از ورودی خواندن به کاراکتری برخوردیم که نمی‌توانستیم با آن به استیت بعدی‌ای برویم، DFA خطا می‌دهد.
\\\\\\
۲) کد به زبان پایتون در زیر آمده است:
\begin{latin}
	\begin{verbatim}
		import re  
		 
		while True:
		    s = input()
		    if re.match("^(-|\+)?(0\d*)$",s):
		        print(int(s,8), "octate")
		    if re.match("^(-|\+)?0(x|X)([0-9a-fA-F]+)$",s):
		        print((int(s,0)), "hex")
		    if re.match("^(-|\+)?[1-9]\d+$",s):
		        print((int(s)), "decimal")
	\end{verbatim}
\end{latin}