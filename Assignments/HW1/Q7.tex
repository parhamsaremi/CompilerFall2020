%نام و نام خانوادگی:
%شماره دانشجویی: 
\مسئله{\lr{DFA} به \lr{NFA}}

\پاسخ{ 
با دقت در $DFA$ صورت سوال متوجه می‌شویم که از $q_i$ در صورت دریافت ۰ به 
$q_{2*i\%8}$
می‌رویم و در صورت دریافت ۱ به 
$q_{(2*i+1)\%8}$ 
می‌رویم بنابراین عملا این $DFA$ دارد باقیمانده یک رشته باینری به عددی ۸ را حساب می‌کند (سه حرف آخر) و در صورتی که باقی‌مانده برابر با ۴، ۵، ۶ یا ۷ بود آن رشته را قبول می‌کند.\\
حالت‌های سه کاراکتر آخر برای قبول شدن را به شکل زیر داریم:
\begin{latin}
\begin{itemize}
	\item $q_4$: 100
	\item $q_5$: 101
	\item $q_6$: 110
	\item $q_7$: 111
\end{itemize}
\end{latin}
بنابراین برای سه رشته آخر داریم:
\begin{center}
\begin{latin}
$1(1|0)\{2\}$
\end{latin} 
\end{center}
در نتیجه عبارت منظمی که این $DFA$ نشان می‌دهد برابر است با: 
\begin{center}
\begin{latin}
$(0|1)*1(0|1)\{2\}$
\end{latin}
\end{center} 
که $NFA$ معادل آن با استفاده از ۴ حالت در زیر قرار گرفته‌است.

\begin{latin}
\begin{center}
	\begin{tikzpicture}
		[->,
		>=stealth,
		node distance=3cm,
		every state/.style={thick, fill=white!10},
		initial text=$ $,
		]
		\node[state, initial] (q1) {$q_1$};
		\node[state, right of=q1] (q2) {$q_2$};
		\node[state, right of=q2] (q3) {$q_3$};
		\node[state, accepting, right of=q3] (q4) {$q_4$};
		\draw 
		(q1) edge[loop above] node{0,1} (q1)
		(q1) edge[above] node{1} (q2)
		(q2) edge[above] node{0,1} (q3)
		(q3) edge[above] node{0,1} (q4);
	\end{tikzpicture}
\end{center}
\end{latin}
}
