%نام و نام خانوادگی:
%شماره دانشجویی:
\مسئله{تبدیل \lr{NFA}  به \lr{DFA} }

\پاسخ{}
\\
1.
\\
m تاRمتوالی به عنوان استیت‌هایمان داریم   و از nامین تا m امین آنها به حالت نهایی بایدtransition {{$\epsilon$}} داشته باشیم. این طراحی بیشتر از mتا R را نیز نمیتواند بپذیرد .
برای مثال حالت n=2 , m=4 را در نظر بگیرید:
\\
\\
\newpage
\begin{center}
\begin{tikzpicture}[shorten >=1pt,node distance=2cm,on grid,auto]
\node[state,initial]   (s)                    {$s$};
\node[state]   (R0)      [right of=s]               {$R$};
\node[state]  (R1) [right of=R0]              {$R$};
\node[state]  (R2) [right of=R1]              {$R$};
\node[state]  (R3) [right of=R2]              {$R$};
\node[state,accepting]  (final)  [right of=R3]    {$final$};
\path[->]
(s) edge []  node {{$\epsilon$}}    (R0)
(R0) edge []  node {{$\epsilon$}}    (R1)
(R1) edge []  node {{$\epsilon$}}    (R2)
    edge [bend left=1.5cm]  node {{$\epsilon$}}    (final)
(R2) edge []  node {{$\epsilon$}}    (R3)
    edge [bend left]  node {{$\epsilon$}}    (final)
(R3) edge []  node {{$\epsilon$}}    (final);
\end{tikzpicture}
\end{center}

2.
\\
در کل 6 مسیر aabb را مشخص میکنند:
(همواره در استیت صفر هستیم و از بعد آن استیت ها را به ترتیب ذکر کرده شده‌اند.)
\\
1223
\\
1100
\\
0111
\\
0110
\\
0112
\\
0122\\
که از بین آنها مسیر  1223 در استیت نهایی رشته را تمام میکند و لذا رشته قابل تشخیص است.
\\
\\
حالت کلی :
\\
\begin{center}
\begin{tikzpicture}[shorten >=1pt,node distance=2cm,on grid,auto]
\node[state,initial]   (0)                    {$0$};
\node[state] (1)  [right of=0]    {$1$};
\node[state]           (2)  [right of=1]    {$2$};
\node[state,accepting]           (3)  [right of=2]    {$3$};
\path[->]
(0) edge []  node {a}    (1)
    edge [loop below] node {a,b}    (   )
(1) 	edge []  node {a}    (2)
    edge [loop below] node {a,b}    (   )
(2) edge [above,bend right]  node {{$\epsilon$}}  (0)
 	edge []  node {b}  (3)
 	edge [loop below] node {a,b}    (  );
\end{tikzpicture}
\end{center}
نمونه‌ی یک مسیر قابل قبول :
\begin{center}
\begin{tikzpicture}[shorten >=1pt,node distance=2cm,on grid,auto]
\node[state,initial]   (0)                    {$0$};
\node[state] (1)  [right of=0]    {$1$};
\node[state]           (2)  [right of=1]    {$2$};
\node[state,accepting]           (3)  [right of=2]    {$3$};
\path[->]
(0) edge []  node {a}    (1)
(1) 	edge []  node {a}    (2)
(2) edge []  node {b}  (3)
 	edge [loop below] node {b}    (  );
\end{tikzpicture}
\end{center}

3.
در هر یک از استیتها که باشیم، 2 حالت قابل قبول برای رسیدن به رشته مورد نظر وجود دارد لذا 2 به توان 4 و یا 16 حالت وجود دارد.
که از بین آنها مسیر 10123 در استیت نهایی رشته را تمام میکند و لذا رشته قابل تشخیص است. لازم به ذکر است که به ازای حروف الفبای اصلی 16 حالت داریم اما در کل به دلیل وجود {{$\epsilon$}}ها ، بی‌نهایت حالت اکسپت میشوند.
\\
حالت کلی :
\\
\begin{center}
\begin{tikzpicture}[shorten >=1pt,node distance=2cm,on grid,auto]
\node[state,initial]   (0)                    {$0$};
\node[state] (1)  [right of=0]    {$1$};
\node[state]           (2)  [right of=1]    {$2$};
\node[state,accepting]           (3)  [right of=2]    {$3$};
\path[->]
(0) edge []  node {a}    (1)
    edge [bend left]  node {{$\epsilon$}}  (3)
(1) edge [bend left]  node {{$\epsilon$}}    (0)
	edge []  node {b}    (2)
(2) edge [bend left]  node {{$\epsilon$}}  (1)
 	edge []  node {b}  (3)
(3) edge [bend left]  node {{$\epsilon$}}  (2)
	edge [bend left=2cm]  node {a}  (0);
\end{tikzpicture}
\end{center}

نمونه‌ی یک مسیر قابل قبول:
\begin{center}
\begin{tikzpicture}[shorten >=1pt,node distance=2cm,on grid,auto]
\node[state,initial]   (0)                    {$0$};
\node[state] (1)  [right of=0]    {$1$};
\node[state]           (2)  [right of=1]    {$2$};
\node[state,accepting]           (3)  [right of=2]    {$3$};
\path[->]
(0) edge []  node {a}    (1)
    edge []  node {a}    (1)
(1) edge [bend left]  node {{$\epsilon$}}  (0)
	edge []  node {b}    (2)
(2)
 	edge []  node {b}  (3);
\end{tikzpicture}
\end{center}