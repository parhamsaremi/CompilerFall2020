%نام و نام خانوادگی:
%شماره دانشجویی: 
\مسئله{تبدیل \lr{NFA}  به \lr{DFA} }

\پاسخ{}
\\
1.
\\
در کل m+n+1 استیت در نظر میگیریم. از استیت m به بعد، با اپسیلون تمام حالات موجود را به حالت نهایی برده و اکسپت میکینم،  سپس برای حالاتی هم که از n تا بیشتر بودند، یک استیت دیگر داریم که ازین به بعد تمام حروف الفبایی که میگیریم را در آن استیت میبریم که قابل قبول نباشد.
تا پیش از استیت m را نیز هر 
بار با دریافت هر حرف الفبایی به استیت بعدی میرویم.
استیتهای میانی با ... مشخص شده اند.
\\
\newpage

\begin{tikzpicture}[shorten >=1pt,node distance=2cm,on grid,auto]
\tikzstyle{every state}=[fill={rgb:black,1;white,10}]
\node[state,initial]   (s1)                    {$s_1$};
\node[state]   (...)      [right of=s1]               {$...$};
\node[state]  (sm) [right of=...]              {$s_(_m_-_1_)$};
\node[state,accepting]   [right of=sm]   (m)                    {$m$};
\node[state,accepting] (n)  [right of=m]    {$n$};
\node[state]           (aftern)  [right of=n]    {$after n$};
\path[->]
(s1) edge [bend left]  node {{$\Sigma$}}    (sm)
(sm) edge []  node {{$\Sigma$}}    (m)
(m) edge []  node {{$\epsilon$}}    (n)
(n) edge []  node {{$\Sigma$}}    (aftern)
(aftern)	edge [loop right] node {{$\Sigma$}}    (   );
\end{tikzpicture}


2.
\\
در کل 6 مسیر aabb را مشخص میکنند:
\\
1223
\\
1100
\\
0111
\\
0110
\\
0112
\\
0122\\
که از بین آنها مسیر زیر 1223 در استیت نهایی رشته را تمام میکند و لذا رشته قابل تشخیص است.
\\
3.
در هر یک از استیتها که باشیم، 2 حالت قابل قبول برای رسیدن به رشته مورد نظر وجود دارد لذا 2 به توان 4 و یا 16 حالت وجود دارد.
