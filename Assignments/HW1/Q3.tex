%نام و نام خانوادگی:
%شماره دانشجویی: 
\مسئله{\lr{NFA} معادل}


\پاسخ{}
\\در هر قسمت اگر مقداری با pattern داده شده مطابقت داشت، آن را انتخاب کرده و در صورتی که حالت بعدی که برای آن در نظر میگیریم طول رشته بیشتری را شامل بشود، حالت در نظر گرفته را آپدیت میکنیم. همچنین میدانیم که عملگر + به معنای تعداد 1 یا بیشتر از آن کارکتر است و عملگر | به معنای کافی بودن وجود یکی از کارکترها میباشد.
\\1. رشته‌ی aaabccabbb
\\
در این حالت aaab نسبت به حالات قبلی دارای بیشترین طول است پس باحالت سوم مطابقت داده میشود و جایگزین حالت قبلی میگردد. سپس دو تا کارکتر c داریم که با 4 مطابقت داده میشوند و بعد نیز abbb را با حالت سوم تطبیق میدهیم. در نهایت رشته خروجی برابر 2442 خواهد بود.
\\
\\2.رشته‌ی cbbbbac
\\
ابتدا به ازای کارکتر نخست 4 را خروجی داده و سپس چهار کارکتر بعدی با حالت سوم مطابقت پیدا میکنند. این چهار کارکتر با حالت چهارم نیز مطابق هستند اما چون افزایش طولی نداریم، پس مقدار جدیدی به خود نمیگیرند. سپس دو کارکتر انتهایی هم به وضوح با حالت دوم و پنجم مطابق شده و خروجی نهایی برابر 4214 خواهد بود.
\\
\\3. رشته‌ی cbabc
\\
در ابتدا و انتها که برای کارکتر c 4 خروجی میدهیم و برای کارکترهای میانی نیز بیشترین طول برابر 3 خواهد بود که برای این حالت باید 2 را خروجی داد. پس در نهایت جواب برابر 424 خواهد شد.