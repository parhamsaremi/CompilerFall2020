\مسئله{}
\پاسخ{}
\\
الف) متغیرهای زنده هر خط به در زیر نمایش داده شده‌اند: (دقت شود که مقادیر خط i ام در پایین، متغیرهای زنده پس از اجرای خط i ام برنامه را نشان می‌دهند. همچنین مقادیر خط ۰ ام و ۱۱ ام به ترتیب مقادیر قبل از شروع اجرای برنامه و مقادیر پس از اجرای خط ۱۱ برنامه را نشان می‌دهند)
\begin{latin}
	\begin{verbatim}
		0. {f, b, e}
		1. {f, a, b, e}
		2. {f, a, b, e}
		3. {f, a, b, e}
		4. {f, c, a, b}
		5. {d, f, c, a, b}
		6. {x, d, f, c, a}
		7. {x, z, d, f, c}
		8. {x, z, d, f}
		9. {x, z, d}
		10. {x, z, d, y}
		11. {x, z, d}
	\end{verbatim}
\end{latin}
ب) با توجه به بخش بالا، باید خطوط کامنت شده در زیر حذف شوند به این خاطر که code dead هستند:
\begin{latin}
	\begin{verbatim}
		1. a = 1 + 2;
		2. b = a + b;
		3. //z = a * 2;
		4. c = b + e;
		5. d = c + b;
		6. x = b + 3;
		7. z = a * 8;
		8. //t = c - 2;
		9. //f = x + f;
		10. y = x - 2;
		11. d = d - y;
	\end{verbatim}
\end{latin}
پس از حذف آن‌ها به کد زیر می‌رسیم:
\begin{latin}
	\begin{verbatim}
		1. a = 1 + 2;
		2. b = a + b;
		3. c = b + e;
		4. d = c + b;
		5. x = b + 3;
		6. z = a * 8;
		7. y = x - 2;
		8. d = d - y;
	\end{verbatim}
\end{latin}
حال براساس propagation copy و folding constant کد را بهینه‌تر میکنیم و به کد زیر می‌رسیم:
\begin{latin}
	\begin{verbatim}
		1. a = 3;
		2. b = 3 + b;
		3. c = b + e;
		4. d = c + b;
		5. x = b;
		6. z = 24;
		7. y = b - 2;
		8. d = d - y;
	\end{verbatim}
\end{latin}
حال یک بار دیگر code dead ها را حذف می‌کنیم تا خط اول هم حذف شود و به کد نهایی زیر می‌رسیم. این بهینه‌ترین کد ممکن هست:
\begin{latin}
	\begin{verbatim}
		1. b = 3 + b;
		2. c = b + e;
		3. d = c + b;
		4. x = b;
		5. z = 24;
		6. y = b - 2;
		7. d = d - y;
	\end{verbatim}
\end{latin}
















