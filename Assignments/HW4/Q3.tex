\مسئله{}
\پاسخ{}
فرض می‌کنیم که کد main به صورت زیر باشد:
\begin{latin}
	\begin{verbatim}
		1. int main(){
		2.     C* c = new C();
		3.     c->f(137);
		4. }
	\end{verbatim}
\end{latin}
در این صورت کد TAC تولیدشده به صورت زیر خواهد بود:
\begin{latin}
	\begin{verbatim}
		1. main:
		2. BeginFunc 16;
		3. _t0 = 16;
		4. PushParam _t0;
		5. c = LCall _Alloc;
		6. PopParam 4;
		7. _t1 = C_share_A
		8. *c = _t1
		9. _t1 = C_share_B
		10. _t2 = c + 4
		11. *_t2 = _t1
		12. _t2 = c + 8
		13. *_t2 = int_default
		14. _t2 = c + 12
		15. *_t2 = int_default
		16. _t1 = *c
		17. _t1 = _t1 + 8
		18. _t2 = *_t1
		19. _t1 = 137
		20. PushParam c
		21. PushParam _t1
		22. ACall _t2
		23. PopParams 8
		24. EndFunc;
		.
		.
		.
		25. C.f:
		26. BeginFunc ...;
		27. ...
		28. EndFunc;
	\end{verbatim}
\end{latin}
حال TAC را توضیح می‌دهیم.
\\
در خطوط ۳ تا ۶ فضای لازم برای object را allocate می‌کنیم.
\\
در خطوط ۷ و ۸ اشاره‌گر به vtable مربوط به C-share-A را ایجاد و مقداردهی می‌کنیم.
\\
در خطوط ۹ تا ۱۱ همین کار را برای C-share-B انجام می‌دهیم.
\\
در خطوط ۱۲ و ۱۳ به متغیر a در کلاس C فضا اختصاص می‌دهیم و به آن مقدار پیش‌فرض را می‌دهیم.
\\
در خطوط ۱۴ و ۱۵ همین کار را برای متغیر b از کلاس C می‌کنیم.
\\
در خطوط ۱۶ تا ۱۹ تابع f را load می‌کنیم و آدرس آن را در t2 می‌ریزیم. در خطوط ۲۰ تا ۲۳ تابع را صدا می‌زنیم.
\\
خطوط ۲۵ تا ۲۸ هم مکان فرضی کد تابع f را نشان می‌دهد که البته با آن کاری نداریم چون کد به طور مشخص در صورت سوال معلوم نشده است.





