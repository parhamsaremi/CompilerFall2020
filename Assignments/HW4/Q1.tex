%نام و نام خانوادگی:
%شماره دانشجویی: 
\مسئله{طراحی \lr{DFA}}

\پاسخ{}

علی‌رغم این‌که
تعداد
ورودی‌ها در زمان
اجرا مشخص می‌شود ، اما چون 
تعداد ورودی‌های تابع ثابت نیست،
لازم است تابعی تعریف کنیم تا به ازای هر بار فراخوانی در کد،
کد مربوط به فراخوانی را بسازد.
چرا که این فراخوانی‌ها مشابه بوده و تفاوتشان در تعداد پارامترهایی که باید در استک قرار بگیرند خواهد بود.
به این صورت که در هنگام فراخوانی داخل چنین توابعی ، لازم است که تعداد پارامترها را نیز درون استک بگذاریم تا آن برنامه با داشتن تعداد دقیق ورودی‌هایش 
بتواند
از همه‌ی 
آن‌ها استفاده کند.
در غیر این صورت ممکن متوجه نشود که چند ورودی در هر حالت به تابع داده شده است و لذا امکان ایجاد مشکل در هنگام فراخوانی پارامترها وجود دارد.
برای این که از ابتدا بتوانیم این تعداد را در نظر بگیریم ، فرض می‌کنیم که این مقدار در سر استک قرار دارد و با اولین pop
به دست می‌آید.
سپس پارامترهای ورودی را نیز به ترتیب داخل این استک قرار می‌دهیم.
در واقع این چنین عمل کردیم که خود تابع 
cgen
،
تابعی همراه با لیستی از پارامترهایش دریافت کند و کد لازم را برای فراخواندن آن بسازد.
نحوه‌ی این پیاده‌سازی به این شکل است
:
\\
لازم به ذکر است که فرض کرده‌ایم هر پارامتر 4 بایت است.
\\
\begin{latin}
\textbf{cgen}(id(list of expressions))=
\big\{ 
\\
Let numberOfExpressions be the number of expressions in the list.
\\
enumerate expressions from 1 to numberOfExpressions.
\\
_{ }
{Let} {$ $} t_{1}=\textbf{cgen}(expr_{1})
\\
Let {$ $} t_{2}=\textbf{cgen}(expr_{2})
\\
Let {$ $} t_{3}=\textbf{cgen}(expr_{3})
\\
.
\\
.
\\
.
\\
Let {$ $} t_{n}=\textbf{cgen}(expr_{n})
\\
Emit(PushParam {$ $} t_{n});
\\
Emit(PushParam {$ $} t_{n-1});
\\
Emit(PushParam {$ $} t_{n-2});
\\
.
\\
.
\\
.
\\
Emit(PushParam {$ $} t_{1});
\\
Emit(PushParam {$ $} numberOfExpressions);
\\
Emit(ACall {$ $} id);
\\
Emit(PopParam {$ $} 4*(n+1));
\big\}
\end{latin}
\newpage
