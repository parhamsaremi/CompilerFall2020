%نام و نام خانوادگی:
%شماره دانشجویی: 
\مسئله{عبارت آرمانی}
\پاسخ{}
	
\begin{enumerate}
	\item
	با استفاده از گراف نحو زیر عبارت‌های شامل جمع و تفریق را تشخیص می‌دهیم که در آن از یک پارس استک استفاده می‌شود: 
	\begin{latin}
		E:
		\begin{center}
			\begin{tikzpicture}
				[->,
				>=stealth,
				node distance=3cm,
				every state/.style={thick, fill=white!10},
				initial text=$ $,
				]
				\node[state] (q1) {$1$};
				\node[state, right of=q1, accepting] (q2) {$2$};
				\draw 
				(q1) edge[above] node{$id$} (q2)
				(q2) edge[above, bend right = 2cm] node{$+-$} (q1);
			\end{tikzpicture}
		\end{center}
	\end{latin}
حال برای تشخیص دادن آرمانی بودن از یک استک دیگر استفاده می‌کنیم به این شکل که هر زمان یک مثبت دیدیدم آن را به استک اضافه می‌کنیم و هر وقت منفی دیدیم یک مثبت را از استک pop می‌کنیم و اگر زمانی هنگام pop کردن استک خالی بود به این معناست که تا اینجای عبارت تعداد منفی‌ها از مثبت‌ها بیشتر شده است و عبارت آرمانی نمی‌باشد و خطا برمی‌گردانیم.
	\item  
	برای اینکه بدون استفاده از استک اضافی بتوانیم آرمانی بودن را تشخیص دهیم از گراف‌های نحوی زیر استفاده می‌کنیم این گراف‌ها با ایده گرففتن از گرامر زیر بدست آمده‌اند:
	\begin{latin}
		\textcolor{red}{E} $\rightarrow$ \textcolor{blue}{id}\textcolor{red}{V}\\
		\textcolor{red}{V} $\rightarrow$
		 \textcolor{blue}{+id}\textcolor{red}{V}\textcolor{blue}{-id}\textcolor{red}{V'}|\textcolor{red}{V'}\\
		\textcolor{red}{V'} $\rightarrow$
		 \textcolor{blue}{+id}\textcolor{red}{V'}|\textcolor{blue}{$\epsilon$}
	\end{latin}
که ایده به این صورت می‌باشد که اگر اولین + را با آخرین - در نظر بگیریم بین این دو، یک عبارت با گرامر V دیگر بدست می‌آید همچنین بعد از آخرین - ممکن است چند + داشته باشیم و یا هیچی نداشته باشیم که علت اضافه کردن 
\lr{V'} به این دلیل می‌باشد هر چند از \lr{V'} در گراف استفاده نشده و با متصل کردن حالت ۸ام به حالت ۴ام پیاده‌سازی شده است.
	\begin{latin}
		E:
		\begin{center}
			\begin{tikzpicture}
				[->,
				>=stealth,
				node distance=3cm,
				every state/.style={thick, fill=white!10},
				initial text=$ $,
				]
				\node[state] (q1) {$1$};
				\node[state, right of=q1, accepting] (q2) {$2$};
				\node[state, right of=q2,accepting] (q3) {$3$};
				\draw 
				(q1) edge[above] node{$id$} (q2)
				(q2) edge[above] node{$V$} (q3);
			\end{tikzpicture}
		\end{center}
		V:
		\begin{center}
			\begin{tikzpicture}
				[->,
				>=stealth,
				node distance=3cm,
				every state/.style={thick, fill=white!10},
				initial text=$ $,
				]
				\node[state, accepting] (q1) {$4$};
				\node[state, right of=q1] (q2) {$5$};
				\node[state, right of=q2,accepting] (q3) {$6$};
				\node[state, right of=q3] (q4) {$7$};
				\node[state, right of=q4] (q5) {$8$};				
				\draw 
				(q1) edge[above] node{$+$} (q2)
				(q2) edge[above] node{$id$} (q3)
				(q3) edge[above] node{$V$} (q4)
				(q4) edge[above] node{$-$} (q5)
				(q5) edge[below, bend left=1cm] node{$id$} (q1);
				
			\end{tikzpicture}
		\end{center}
	\end{latin}
به عنوان مثال برای \lr{id+id-id+id} ابتدا از حالت ۱ به حالت ۲ می‌رویم سپس با دیدن + داخل گراف  V می‌رویم و از حالت ۴ به حالت ۵ رفته و با دریافت id به حالت ۶ می‌رویم حال با دریافت منفی با توجه به اینکه یالی با این علامت نداریم به داخل گراف V می‌رویم و در حالت ۴ قرار می‌گیریم با توجه به اینکه در این حالت نیز یالی به علامت - نداریم و در یک حالت پایانی می‌باشیم از این گراف خارج شده و با توجه به اینکه حالت بالای استک برابر با شماره ۶ می‌باشد و خروج بدون خطا بوده است به حالت شماره ۷ می‌رویم از آنجا با دریافت - به حالت ۸ و با دریافت id به حالت ۴ رفته  از آنجا با دریافت مثبت و id به ۵ و ۶ رفته و با توجه به تمام شدن عبارت و قرار داشتن در حالت پایانی با موفقیت برنامه تمام می‌شود.\\
اگر رشته‌ی بالا \lr{id+id-id-id} می‌بود با دریافت - دوم با توجه به اینکه در حالت ۴ بوده و عدد بالای استک هم حالت ۲ می‌باشد به حالت پایانی می‌رسیم در حالی که رشته تمام نشده و با خطا مواجه می‌شویم.
\end{enumerate} 