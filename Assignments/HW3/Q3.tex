\مسئله{\lr{Syntax Graph}}
\حل{}

گراف‌ها به صورت زیر می‌باشند در ابتدا گراف E را داریم:
\begin{latin}
	\begin{center}
		\begin{tikzpicture}
			[->,
			>=stealth,
			node distance=3cm,
			every state/.style={thick, fill=white!10},
			initial text=$ $,
			]
			\node[state] (S1){$S1$};
			\node[state,accepting, right =of S1] (S2){$S2$};
			\node[state, right =of S2] (S3){$S3$};
			
			\path (S1) edge [] node[above] {$T$} (S2);
			\path (S2) edge [] node[above] {$+$} (S3);
			\path (S3) edge [bend right] node[above] {$T/@add$} (S2);
		\end{tikzpicture}
	\end{center}
\end{latin}
\begin{latin}
	T:
\end{latin}
\begin{latin}
	\begin{center}
		\begin{tikzpicture}
			[->,
			>=stealth,
			node distance=3cm,
			every state/.style={thick, fill=white!10},
			initial text=$ $,
			]
			\node[state] (S1){$S5$};
			\node[state,accepting, right =of S1] (S2){$S6$};
			\node[state, right =of S2] (S3){$S7$};
			
			\path (S1) edge [] node[above] {$F$} (S2);
			\path (S2) edge [] node[above] {$*$} (S3);
			\path (S3) edge [bend right] node[above] {$F/@mult$} (S2);
		\end{tikzpicture}
	\end{center}
\end{latin}
\begin{latin}
	F:
\end{latin}
\begin{latin}
	\begin{center}
		\begin{tikzpicture}
			[->,
			>=stealth,
			node distance=2cm,
			every state/.style={thick, fill=white!10},
			initial text=$ $,
			]
			\node[state] (S1){$S8$};
			\node[state,accepting, right =of S1] (S2){$S9$};
			\node[state, below =of S1] (S3){$S10$};
			\node[state, right =of S3] (S4) {$S11$};
			
			\path (S1) edge [] node[below] {$id/@push$} (S2);
			\path (S1) edge [] node[left] {$($} (S3);
			\path (S3) edge [] node[below] {$E$} (S4);
			\path (S4) edge [] node[right] {$)$} (S2);
		\end{tikzpicture}
	\end{center}
\end{latin}
حال به توضیح دادن نحوه‌ی جلو رفتن و پر شدن استک می‌پردازیم رشته‌ی داده شده به $id+id$ تبدیل می‌شود حال ابتدا id اول خوانده می‌شود با توجه به اینکه در گره S1 می‌باشیم و هیچ یالی با برچسب id نداریم به داخل گراف T می‌رویم حال در گره S5 می‌باشیم و از آنجا نیز با توجه به اینکه توکن مورد نظر id می‌باشد و چنین یالی نداریم به داخل گراف F می‌رویم حال در گره S8 قرار داریم با توجه به اینکه توکنی که داریم id می‌باشد توسط یال آن به S9 می رویم و عملیات پوش را انجام می‌دهیم و a را به استک اضافه می‌کنیم. در این حالت استک به صورت زیر می‌باشد:
\begin{latin}
	\begin{center}
		\begin{tabular}{|c|}
			\hline
			a\\
			\hline
			
		\end{tabular}
	\end{center}
\end{latin}
با دیدن a و پایان گراف F از این گراف خارج شده و به گراف T بر می‌گردیم و حال در گره S6 قرار داریم اکنون توکنی که در حال پردازش هستیم برابر با + می‌باشد از آنجایی که چنین یالی وجود ندارد و در حالت پایانی گراف می‌باشیم به گراف E باز می‌گردیم و در S2 قرار می‌گیریم. حال با دیدن + به گره S3 می‌رویم و از آنجا با دیدن توکن id دوباره مانند قبل عمل می‌کنیم وقتی دوباره به S8 رسیدیم و b را به استک اضافه کنیم استک به صورت زیر بدست می‌آید:
\begin{latin}
	\begin{center}
		\begin{tabular}{|c|}
			\hline
			a\\
			\hline
			b\\
			\hline
		\end{tabular}
	\end{center}
\end{latin}
حال با توجه به اینکه رشته پایان یافته از گراف F خارج می‌شویم و از گراف T هم خارج می‌شویم حال در حالت S3 قرار داریم از اینجا نیز با توجه به اینکه گراف T را تمام کرده‌ایم با یال مربوط به آن به S2 می‌رویم و در این یال عملیات @add نیز داریم حال دو عنصر آخر را از استک پاپ کرده و جمع می‌کنیم و در استک پوش می‌کنیم، استک به صورت زیر بدست می‌آید:
\begin{latin}
	\begin{center}
		\begin{tabular}{|c|}
			\hline
			a+b\\
			\hline
			
		\end{tabular}
	\end{center}
\end{latin}


