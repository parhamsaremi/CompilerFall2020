%نام و نام خانوادگی:
%شماره دانشجویی:
\مسئله{پرانتزگذاری معتبر}
\پاسخ{
\begin{enumerate}
	\item 
	گرامر با توجه به نداشتن برخورد‌های first/first یا first/follow یک گرامر \lr{LL(1)} می‌باشد حال جدول پارس را می‌کشیم و با توجه به اینکه در آن تداخلی وجود ندارد از این امر مطمئن می‌شویم ابتدا جدول first و foloow را می‌کشیم:
	\begin{latin}
		\begin{center}
			\begin{tabular}{ |c|c|c| } 
				\hline
				& first & follow \\ 
				\hline
				P & $($ & $(,),\$$ \\ 
				\hline
				S & $\epsilon,($ & $),\$$ \\ 
				\hline
			\end{tabular}
		\end{center} 
	\end{latin}
	که با توجه به جدول بالا، جدول پارس به صورت زیر بدست می‌آید:
	\begin{latin}
		\begin{center}
			\begin{tabular}{ |c|c|c|c| } 
				\hline
				& $($ & $)$ & $\$$\\ 
				\hline
				P & $P\rightarrow (S)$ & &\\ 
				\hline
				S & $S\rightarrow PS$ & $S\rightarrow\epsilon$&$S\rightarrow\epsilon$ \\ 
				\hline
			\end{tabular}
		\end{center} 
	\end{latin}
	\item 
	دو مورد خطا ممکن است رخ دهد، مورد اول این است که در بالای استک یک ترمینال باشد که با توکنی که در ابتدای رشته می‌باشد تطابق نداشته باشد مثالی که برای این خطا می‌توان زد به صورت زیر می‌‌باشد:\\
	اگر ورودی را
	\lr{())} 
	بدهیم روند پارس به صورت زیر خواهد بود:\\
	\begin{latin}
		S\$ $\rightarrow$ PS\$ $\rightarrow$ (S)S\$ $\rightarrow$ S)S\$ $\rightarrow$ )S\$
		$\rightarrow$ S\$ $\rightarrow$\$ \\
	\end{latin}
	
	که در این مرحله در بالای استک یک ترمینال داریم که با ( مطابقت ندارد بنابراین به این خطا برخورد می‌کنیم.\\
	خطای مورد دوم به این صورت می‌باشد که یک عبارت 
	\lr{non terminal} در بالای استک می‌باشد و با کاراکتری که در ابتدای رشته وجود دارد نمی‌تواند یک عبارت را پیشبینی و جایگزین کند. از آنجایی که برای رسیدن به این حالت باید در بالای استک P باشد و در ابتدای رشته یا \$ باشد و یا ) و رخ دادن چنین حالتی ممکن نمی‌باشد برای این قسممت مثالی زده نشده.
\end{enumerate}
}