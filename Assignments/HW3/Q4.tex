%نام و نام خانوادگی:
%شماره دانشجویی: 
\مسئله{}
\پاسخ{ }
\\
\\
۱)
 مزیت: کامپایلر $single-pass$ چون تنها یکبار کل کد را می‌خواند سریع‌تر عمل کامپایل را انجام می‌دهد ولی $multi-pass$ چون چندبار می‌خواند زمان بیش‌تری نیاز دارد.
\\
\\
عیب: کاپایلر $single-pass$ چون فقط یکبار کد را می‌خواند نمی‌تواند اطلاعات زیادی از کد بدست آورد و ممکن است برنامه خروجی به اندازه کافی بهینه نشود، همچنین به همین خاطر مجبور است قوانین سخت‌گیرانه‌ای روی $syntax$ بگذارد تا بتواند صرفا با یکبار خواندن کد، معنای آن را بفهمد و آن را کامپایل کند اما در $multi-pass$ این طور نیست و چنین مشکلاتی نداریم.
\\
\\
۲)
مزیت: $static-type-checking$ چون در زمان کامپایل انجام می‌شود باعث می‌شود که برنامه خروجی سریع‌تر باشد، چرا که دیگر لازم نیست مانند $dynamic-type-checking$ این کار را در زمان اجرای برنامه انجام دهیم و برنامه کند شود.

عیب: به این خاطر که در $static-type-checking$ باید نوع هر متغیر یا عبارت در زمان کامپایل معلوم باشد، کامپایلر مجبور است محدودیات بیش‌تری روی $syntax$ بگذارد تا بتواند در زمان کامپایل همه انواع را به درستی چک کند. در مقابل اما $dynamic-type-checking$ چون می‌تواند در زمان اجرا انواع را بررسی کند که در آن زمان، روند اجرای برنامه، اطلاعات بیش‌تری را در مورد متغیرها و عبارات معلوم کرده است، کم‌تر به این نیاز دارد که $syntax$ را محدود کند و دست برنامه‌نویس را بازتر می‌گذارد، در نتیجه برنامه معمولا فشرده‌تر هم می‌شود.
\\
\\
دقت شود چون صورت سوال گفته بود که از هرکدام یک مورد بگوییم من این طور برداشت کردم که باید یک مزیت و یک عیب بنویسم و چیز بیش‌تر لازم نیست.
\\